\documentclass[]{article}
\usepackage{lmodern}
\usepackage{amssymb,amsmath}
\usepackage{ifxetex,ifluatex}
\usepackage{fixltx2e} % provides \textsubscript
\ifnum 0\ifxetex 1\fi\ifluatex 1\fi=0 % if pdftex
  \usepackage[T1]{fontenc}
  \usepackage[utf8]{inputenc}
\else % if luatex or xelatex
  \ifxetex
    \usepackage{mathspec}
  \else
    \usepackage{fontspec}
  \fi
  \defaultfontfeatures{Ligatures=TeX,Scale=MatchLowercase}
\fi
% use upquote if available, for straight quotes in verbatim environments
\IfFileExists{upquote.sty}{\usepackage{upquote}}{}
% use microtype if available
\IfFileExists{microtype.sty}{%
\usepackage{microtype}
\UseMicrotypeSet[protrusion]{basicmath} % disable protrusion for tt fonts
}{}
\usepackage[margin=1in]{geometry}
\usepackage{hyperref}
\hypersetup{unicode=true,
            pdftitle={SWUE Projekt - Aufgabe 7},
            pdfauthor={Tobias, Julia, Markus},
            pdfborder={0 0 0},
            breaklinks=true}
\urlstyle{same}  % don't use monospace font for urls
\usepackage{color}
\usepackage{fancyvrb}
\newcommand{\VerbBar}{|}
\newcommand{\VERB}{\Verb[commandchars=\\\{\}]}
\DefineVerbatimEnvironment{Highlighting}{Verbatim}{commandchars=\\\{\}}
% Add ',fontsize=\small' for more characters per line
\usepackage{framed}
\definecolor{shadecolor}{RGB}{248,248,248}
\newenvironment{Shaded}{\begin{snugshade}}{\end{snugshade}}
\newcommand{\KeywordTok}[1]{\textcolor[rgb]{0.13,0.29,0.53}{\textbf{#1}}}
\newcommand{\DataTypeTok}[1]{\textcolor[rgb]{0.13,0.29,0.53}{#1}}
\newcommand{\DecValTok}[1]{\textcolor[rgb]{0.00,0.00,0.81}{#1}}
\newcommand{\BaseNTok}[1]{\textcolor[rgb]{0.00,0.00,0.81}{#1}}
\newcommand{\FloatTok}[1]{\textcolor[rgb]{0.00,0.00,0.81}{#1}}
\newcommand{\ConstantTok}[1]{\textcolor[rgb]{0.00,0.00,0.00}{#1}}
\newcommand{\CharTok}[1]{\textcolor[rgb]{0.31,0.60,0.02}{#1}}
\newcommand{\SpecialCharTok}[1]{\textcolor[rgb]{0.00,0.00,0.00}{#1}}
\newcommand{\StringTok}[1]{\textcolor[rgb]{0.31,0.60,0.02}{#1}}
\newcommand{\VerbatimStringTok}[1]{\textcolor[rgb]{0.31,0.60,0.02}{#1}}
\newcommand{\SpecialStringTok}[1]{\textcolor[rgb]{0.31,0.60,0.02}{#1}}
\newcommand{\ImportTok}[1]{#1}
\newcommand{\CommentTok}[1]{\textcolor[rgb]{0.56,0.35,0.01}{\textit{#1}}}
\newcommand{\DocumentationTok}[1]{\textcolor[rgb]{0.56,0.35,0.01}{\textbf{\textit{#1}}}}
\newcommand{\AnnotationTok}[1]{\textcolor[rgb]{0.56,0.35,0.01}{\textbf{\textit{#1}}}}
\newcommand{\CommentVarTok}[1]{\textcolor[rgb]{0.56,0.35,0.01}{\textbf{\textit{#1}}}}
\newcommand{\OtherTok}[1]{\textcolor[rgb]{0.56,0.35,0.01}{#1}}
\newcommand{\FunctionTok}[1]{\textcolor[rgb]{0.00,0.00,0.00}{#1}}
\newcommand{\VariableTok}[1]{\textcolor[rgb]{0.00,0.00,0.00}{#1}}
\newcommand{\ControlFlowTok}[1]{\textcolor[rgb]{0.13,0.29,0.53}{\textbf{#1}}}
\newcommand{\OperatorTok}[1]{\textcolor[rgb]{0.81,0.36,0.00}{\textbf{#1}}}
\newcommand{\BuiltInTok}[1]{#1}
\newcommand{\ExtensionTok}[1]{#1}
\newcommand{\PreprocessorTok}[1]{\textcolor[rgb]{0.56,0.35,0.01}{\textit{#1}}}
\newcommand{\AttributeTok}[1]{\textcolor[rgb]{0.77,0.63,0.00}{#1}}
\newcommand{\RegionMarkerTok}[1]{#1}
\newcommand{\InformationTok}[1]{\textcolor[rgb]{0.56,0.35,0.01}{\textbf{\textit{#1}}}}
\newcommand{\WarningTok}[1]{\textcolor[rgb]{0.56,0.35,0.01}{\textbf{\textit{#1}}}}
\newcommand{\AlertTok}[1]{\textcolor[rgb]{0.94,0.16,0.16}{#1}}
\newcommand{\ErrorTok}[1]{\textcolor[rgb]{0.64,0.00,0.00}{\textbf{#1}}}
\newcommand{\NormalTok}[1]{#1}
\usepackage{graphicx,grffile}
\makeatletter
\def\maxwidth{\ifdim\Gin@nat@width>\linewidth\linewidth\else\Gin@nat@width\fi}
\def\maxheight{\ifdim\Gin@nat@height>\textheight\textheight\else\Gin@nat@height\fi}
\makeatother
% Scale images if necessary, so that they will not overflow the page
% margins by default, and it is still possible to overwrite the defaults
% using explicit options in \includegraphics[width, height, ...]{}
\setkeys{Gin}{width=\maxwidth,height=\maxheight,keepaspectratio}
\IfFileExists{parskip.sty}{%
\usepackage{parskip}
}{% else
\setlength{\parindent}{0pt}
\setlength{\parskip}{6pt plus 2pt minus 1pt}
}
\setlength{\emergencystretch}{3em}  % prevent overfull lines
\providecommand{\tightlist}{%
  \setlength{\itemsep}{0pt}\setlength{\parskip}{0pt}}
\setcounter{secnumdepth}{0}
% Redefines (sub)paragraphs to behave more like sections
\ifx\paragraph\undefined\else
\let\oldparagraph\paragraph
\renewcommand{\paragraph}[1]{\oldparagraph{#1}\mbox{}}
\fi
\ifx\subparagraph\undefined\else
\let\oldsubparagraph\subparagraph
\renewcommand{\subparagraph}[1]{\oldsubparagraph{#1}\mbox{}}
\fi

%%% Use protect on footnotes to avoid problems with footnotes in titles
\let\rmarkdownfootnote\footnote%
\def\footnote{\protect\rmarkdownfootnote}

%%% Change title format to be more compact
\usepackage{titling}

% Create subtitle command for use in maketitle
\newcommand{\subtitle}[1]{
  \posttitle{
    \begin{center}\large#1\end{center}
    }
}

\setlength{\droptitle}{-2em}

  \title{SWUE Projekt - Aufgabe 7}
    \pretitle{\vspace{\droptitle}\centering\huge}
  \posttitle{\par}
    \author{Tobias, Julia, Markus}
    \preauthor{\centering\large\emph}
  \postauthor{\par}
    \date{}
    \predate{}\postdate{}
  
\usepackage{mathbb}
\usepackage{amsfonts}

\begin{document}
\maketitle

\includegraphics{projekt_files/figure-latex/unnamed-chunk-8-1.pdf}

ptratio: Schüler-Lehrer-Verhältnis nach Stadt. Durch die density
Funktion (schwarze Linie) kann man sehen und durch die Schiefe kann man
berechnen, dass die Verteilungsform eher linksschief ist (Schiefe
\textless{} 0) Anfangs angenommen es ist normalverteilt, aber man sieht,
dass das nicht so ist. blaue Linie = rnorm density; rote Linie = dnorm
Die Verteilung hat 2 Peaks: bei 14 bis 15 und 19 bis 20.5. 5 (bzw. 6)
Zahlen Zusammenfassung:

\begin{Shaded}
\begin{Highlighting}[]
  \KeywordTok{summary}\NormalTok{(Boston}\OperatorTok{$}\NormalTok{ptratio)}
\end{Highlighting}
\end{Shaded}

\begin{verbatim}
##    Min. 1st Qu.  Median    Mean 3rd Qu.    Max. 
##   12.60   17.40   19.05   18.46   20.20   22.00
\end{verbatim}

\includegraphics{projekt_files/figure-latex/unnamed-chunk-10-1.pdf}

black: 1000 (Bk - 0,63) \^{} 2 wobei Bk der Schwarzanteil der Stadt ist.
Verteilung ist linksschief (Schiefe kleiner 0) und J-förmig.

\begin{Shaded}
\begin{Highlighting}[]
  \KeywordTok{skewness}\NormalTok{(Boston}\OperatorTok{$}\NormalTok{black)}
\end{Highlighting}
\end{Shaded}

\begin{verbatim}
## [1] -2.881798
\end{verbatim}

Annäherung an Paetro-Verteilung mit Intervall \([0, \infty]\) Die
Paetro-Verteilung (grüne Linie) hat hier die meisten Vorkommen bei den
höheren Werten Berechnung des Paramters \(\hat \xi\):

\begin{Shaded}
\begin{Highlighting}[]
  \KeywordTok{min}\NormalTok{(Boston}\OperatorTok{$}\NormalTok{black)}
\end{Highlighting}
\end{Shaded}

\begin{verbatim}
## [1] 0.32
\end{verbatim}

\includegraphics{projekt_files/figure-latex/unnamed-chunk-13-1.pdf}
\includegraphics{projekt_files/figure-latex/unnamed-chunk-14-1.pdf}

lstat: Prozentanteil der Bevölkerung mit niedriger Position in der
sozialen Hierarchie. Dh. schlechte menschliche Lebensumstände (z.B.:
wenig Bildung, keinen Schulabschluss, keine Ausbildung oder Studium,
geringes Einkommen, Migrationshintergrund, \ldots{}) Annahme: ptratio
ist normalverteilt die blaue und rote Linie zeigen, dass die Annahme
stimmt blaue Linie=rnorm density Parameter \(\mu\) und \(\sigma\):

\begin{Shaded}
\begin{Highlighting}[]
  \KeywordTok{mean}\NormalTok{(Boston}\OperatorTok{$}\NormalTok{lstat)}
\end{Highlighting}
\end{Shaded}

\begin{verbatim}
## [1] 12.65306
\end{verbatim}

\begin{Shaded}
\begin{Highlighting}[]
  \KeywordTok{sd}\NormalTok{(Boston}\OperatorTok{$}\NormalTok{lstat)}
\end{Highlighting}
\end{Shaded}

\begin{verbatim}
## [1] 7.141062
\end{verbatim}

\includegraphics{projekt_files/figure-latex/unnamed-chunk-16-1.pdf}

medv: mittlerer Wert von Wohneigentum in \$ 1000. Annäherung an
Normalverteilung, aber rechtsschief (Schiefe \textgreater{} 0) und
leptokurtisch (steilgipfelig) (Kurtosis \textgreater{} 3).

\begin{Shaded}
\begin{Highlighting}[]
  \KeywordTok{skewness}\NormalTok{(Boston}\OperatorTok{$}\NormalTok{medv)}
\end{Highlighting}
\end{Shaded}

\begin{verbatim}
## [1] 1.104811
\end{verbatim}

\begin{Shaded}
\begin{Highlighting}[]
  \KeywordTok{kurtosis}\NormalTok{(Boston}\OperatorTok{$}\NormalTok{medv)}
\end{Highlighting}
\end{Shaded}

\begin{verbatim}
## [1] 4.468629
\end{verbatim}

rote Linie = Normalverteilung, blaue Linie = Normalverteilung von medv
Werten \textless{} 50 (Ausreißer größer/gleich 50 weggeschnitten)

\includegraphics{projekt_files/figure-latex/unnamed-chunk-18-1.pdf}

Crim beschreibt die Verbrechensrate pro Einwohner einer Stadt. Die Größe
lässt sich gut mit einer Pareto-Verteilung beschreiben. Die Parameter
errechnen sich wie folgt:
\[ \hat \xi = min_{1 \leq i \leq n} \, x_i = 0.00632\]
\[ \hat \lambda = \frac{n}{\sum_{i=1}^n log \Big ( \frac{x_i}{\hat \xi}\Big)} =  0.00645905 \]
Die Pareto-Verteilung ereignet sich gut für Datensätze die sich über
mehrere Größenordnungen erstrecken. Das ist hier der Fall, da
\(\frac{max}{min} \approx 14000\) ist. Die Paretto-Verteilund ist jedoch
auf dem Interval \((0, \infty]\) definiert ist, und unsere Daten nur im
Interval \([0, 100]\) auftreten können könnte man denken, dass hier auch
eine Exponentialverteilung mit
\(\tau = \frac{1}{\overline X} = 0.2767382\) zur Beschreibung verwendet
werden kann. Jedoch fällt diese Kurve zu schnell ab und die
Verteilungsfunktion erzeugt bereits bei \(F(17)\) Wahrscheinlichkeiten
jenseits von 99\%.

\includegraphics{projekt_files/figure-latex/unnamed-chunk-19-1.pdf}

Zn beschreibt den Anteil der Wohngrundstücke für Grundstücke mit mehr
als 25.000 sq.ft. Es gibt viele 0-Werte (ca. 73\%). Die restlichen Werte
scheinen keiner konkreten Verteilung zu folgen. Am ehesten würden sich 2
weitere skalierte (Da diese maximal \textasciitilde{}0.27 als Summe
haben dürften) Binomialverteilungen mit den Mittelpunkten 20 und 80
eigenen, da es hier kleinere Spitzen in der Frequenz gibt. Darüber
hinaus könnte es auch eine Gleichverteilung im Intervall \([12.5; 100]\)
sein.

\includegraphics{projekt_files/figure-latex/unnamed-chunk-20-1.pdf}

Indus beschreibt den Anteil der Industriefläche pro Stadt. Wenn man sich
das Datenset ansieht, kann man erkennen, dass Werte öfters vorkommen
(zb. 18.10 kommt 132 Mal vor). Daher nehmen wir an, dass mehrere Vororte
zu einem Industriegebiet zusammengefasst wurden. Für jeden Ort, der Teil
eines Industriegebietes ist, wurde der Wert des gesamten
Industriegebietes verwendet.

Der Anteil der Städte aus der Stichprobe, welche am Charles River
liegen.

\begin{verbatim}
## [1] 0.06916996
\end{verbatim}

Der Anteil der Städte aus der Stichprobe, welche nicht am Charles River
liegen.

\begin{verbatim}
## [1] 0.93083
\end{verbatim}

Chas gibt ab, ob der Vorort an den Charles River angrenzt. Hierbei
handelt es sich um eine Bernoulli Verteilung. Die Wahrscheinlichkeit,
dass das Bernoulli Experiment erfolgreich ist, lässt sich wie folgt
berechnen: \[ \hat p = \frac{\hat p_t}{n} = 0.06916996\] Wobei
\(\hat p_t\) die Anzahl der erfolgreichen Experimente in der Stichprobe
ist.

\includegraphics{projekt_files/figure-latex/unnamed-chunk-23-1.pdf}
\includegraphics{projekt_files/figure-latex/unnamed-chunk-24-1.pdf}

Nox gibt die Konzentration von Stickstoff-Oxiden an. Die Größe scheint
nicht ganz normalverteilt zu sein, da die Dichtefunktion (siehe Plot
links) nicht symmetrisch scheint - sie steigt stärker an, als sie
abfällt. Daher wird versucht, die Größe mit einer logarithmischen
Normalverteilung zu beschreiben. Die transformierte Zufallsvariable
\(Y = log(X)\) ist annähernd normalverteilt (siehe QQPlot). Jedoch ist
das Ergebnis nur marginal besser, als durch eine normale
Normalverteilung. Um die Zufallsvariabel \(Y\) analysieren zu können,
wurde für jedes Element \(x_i\) aus der Stichprobe \(y_i = log(x_1)\)
gesetzt. Der so gewonnene Datensatz repräsentiert nun eine Stichprobe
der Zufallsvariable \(Y\).

\includegraphics{projekt_files/figure-latex/unnamed-chunk-25-1.pdf}

Der rm wert des Datensatzes boston beschreibt die durchschnittliche
Nummer an Räumen von Wohnungen in Boston. Die Verteilung der Daten lässt
sich gut mit einer Normalverteilung (siehe rote line) darstellen. Die
Formel für die Normalverteilung ist:
\[\frac{1}{\sigma\sqrt{2\pi}} * e^{-\frac{1}{2}(\frac{x-\mu}{\sigma})^2}\]
Wobei \(\mu \in \mathbb{R}\) den Mittelwert und \(\sigma > 0\) die
Varianz darstellet. Bei dieser Verteilung sind die Werte: \(\mu\) =
\(6.2846344\) und \(\sigma\) = \(0.7026171\)

\includegraphics{projekt_files/figure-latex/unnamed-chunk-26-1.pdf}

Der age wert des Datensatzes Boston beschreibt den Anteil der
Eigentumswohnungen, die vor 1940 gebaut wurden. Diese Werte stellen
stellen keine konkrete Verteilung dar.

\includegraphics{projekt_files/figure-latex/unnamed-chunk-27-1.pdf}
\includegraphics{projekt_files/figure-latex/unnamed-chunk-28-1.pdf}

Der dis wert des Datensatzes Boston beschreibt den gewichteten
Mittelwert der Entfernungen zu den fünf Bostoner Beschäftigungszentren.
Die Verteilung der Daten lässt sich gut mit einer logarithmischen
Normalverteilung beschreiben (siehe rote Line). Da die Werte auf der
rechten Seite langsamer sinken als bei einer normalverteilung wird
versucht die Kurve mittels einer logarithmischen Normalverteilung zu
approximieren. Die Formel für die Logarithmische Normalverteilungist:
\[ \frac{1}{\sigma x \sqrt{2\pi}}exp(-\frac{(ln(x)-\mu)^2}{2\sigma^2})  \]
Die Transofmierte Zufallsvariable Y = ln(X) folgt ebenfalls einer
Normalverteilung. (siehe qq norm dis) Die Werte dieser Funktion sind:
\(\mu\) = \(1.2034487\) und \(\sigma\) = \(0.5586292\)

\includegraphics{projekt_files/figure-latex/unnamed-chunk-29-1.pdf}

Der rad wert des Datensatzes Boston beschreibt die Zugänglichkeit zu
radialen Autobahnen. Die Kurve auf der linken Seite lässt sich annähernd
mit einer Normalverteilung beschreiben. Die Kurve auf der rechten Seite
ist nur ein Wert und stellt somit keine Verteilung dar. Die Werte der
Normalverteilung sind: \(\mu\) = \(4.4491979\) und \(\sigma\) =
\(1.6328849\).

\includegraphics{projekt_files/figure-latex/unnamed-chunk-30-1.pdf}

Der tax Wert des Datensatzes Boston beschreibt den Immobiliensteuersatz
in voller Höhe pro 10.000 USD. Die Kurve auf der linken Seite lässt sich
annähernd mit einer Normalverteilung beschreiben. Die Kurve auf der
rechten Seite ist nur ein Wert und stellt somit keine Verteilung dar.
Die Werte der Normalverteilung sind: \(\mu\) = 311.927 und \(\sigma\) =
67.828.


\end{document}
