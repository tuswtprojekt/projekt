\documentclass[]{article}
\usepackage{lmodern}
\usepackage{amssymb,amsmath}
\usepackage{ifxetex,ifluatex}
\usepackage{fixltx2e} % provides \textsubscript
\ifnum 0\ifxetex 1\fi\ifluatex 1\fi=0 % if pdftex
  \usepackage[T1]{fontenc}
  \usepackage[utf8]{inputenc}
\else % if luatex or xelatex
  \ifxetex
    \usepackage{mathspec}
  \else
    \usepackage{fontspec}
  \fi
  \defaultfontfeatures{Ligatures=TeX,Scale=MatchLowercase}
\fi
% use upquote if available, for straight quotes in verbatim environments
\IfFileExists{upquote.sty}{\usepackage{upquote}}{}
% use microtype if available
\IfFileExists{microtype.sty}{%
\usepackage{microtype}
\UseMicrotypeSet[protrusion]{basicmath} % disable protrusion for tt fonts
}{}
\usepackage[margin=1in]{geometry}
\usepackage{hyperref}
\hypersetup{unicode=true,
            pdfborder={0 0 0},
            breaklinks=true}
\urlstyle{same}  % don't use monospace font for urls
\usepackage{color}
\usepackage{fancyvrb}
\newcommand{\VerbBar}{|}
\newcommand{\VERB}{\Verb[commandchars=\\\{\}]}
\DefineVerbatimEnvironment{Highlighting}{Verbatim}{commandchars=\\\{\}}
% Add ',fontsize=\small' for more characters per line
\usepackage{framed}
\definecolor{shadecolor}{RGB}{248,248,248}
\newenvironment{Shaded}{\begin{snugshade}}{\end{snugshade}}
\newcommand{\KeywordTok}[1]{\textcolor[rgb]{0.13,0.29,0.53}{\textbf{#1}}}
\newcommand{\DataTypeTok}[1]{\textcolor[rgb]{0.13,0.29,0.53}{#1}}
\newcommand{\DecValTok}[1]{\textcolor[rgb]{0.00,0.00,0.81}{#1}}
\newcommand{\BaseNTok}[1]{\textcolor[rgb]{0.00,0.00,0.81}{#1}}
\newcommand{\FloatTok}[1]{\textcolor[rgb]{0.00,0.00,0.81}{#1}}
\newcommand{\ConstantTok}[1]{\textcolor[rgb]{0.00,0.00,0.00}{#1}}
\newcommand{\CharTok}[1]{\textcolor[rgb]{0.31,0.60,0.02}{#1}}
\newcommand{\SpecialCharTok}[1]{\textcolor[rgb]{0.00,0.00,0.00}{#1}}
\newcommand{\StringTok}[1]{\textcolor[rgb]{0.31,0.60,0.02}{#1}}
\newcommand{\VerbatimStringTok}[1]{\textcolor[rgb]{0.31,0.60,0.02}{#1}}
\newcommand{\SpecialStringTok}[1]{\textcolor[rgb]{0.31,0.60,0.02}{#1}}
\newcommand{\ImportTok}[1]{#1}
\newcommand{\CommentTok}[1]{\textcolor[rgb]{0.56,0.35,0.01}{\textit{#1}}}
\newcommand{\DocumentationTok}[1]{\textcolor[rgb]{0.56,0.35,0.01}{\textbf{\textit{#1}}}}
\newcommand{\AnnotationTok}[1]{\textcolor[rgb]{0.56,0.35,0.01}{\textbf{\textit{#1}}}}
\newcommand{\CommentVarTok}[1]{\textcolor[rgb]{0.56,0.35,0.01}{\textbf{\textit{#1}}}}
\newcommand{\OtherTok}[1]{\textcolor[rgb]{0.56,0.35,0.01}{#1}}
\newcommand{\FunctionTok}[1]{\textcolor[rgb]{0.00,0.00,0.00}{#1}}
\newcommand{\VariableTok}[1]{\textcolor[rgb]{0.00,0.00,0.00}{#1}}
\newcommand{\ControlFlowTok}[1]{\textcolor[rgb]{0.13,0.29,0.53}{\textbf{#1}}}
\newcommand{\OperatorTok}[1]{\textcolor[rgb]{0.81,0.36,0.00}{\textbf{#1}}}
\newcommand{\BuiltInTok}[1]{#1}
\newcommand{\ExtensionTok}[1]{#1}
\newcommand{\PreprocessorTok}[1]{\textcolor[rgb]{0.56,0.35,0.01}{\textit{#1}}}
\newcommand{\AttributeTok}[1]{\textcolor[rgb]{0.77,0.63,0.00}{#1}}
\newcommand{\RegionMarkerTok}[1]{#1}
\newcommand{\InformationTok}[1]{\textcolor[rgb]{0.56,0.35,0.01}{\textbf{\textit{#1}}}}
\newcommand{\WarningTok}[1]{\textcolor[rgb]{0.56,0.35,0.01}{\textbf{\textit{#1}}}}
\newcommand{\AlertTok}[1]{\textcolor[rgb]{0.94,0.16,0.16}{#1}}
\newcommand{\ErrorTok}[1]{\textcolor[rgb]{0.64,0.00,0.00}{\textbf{#1}}}
\newcommand{\NormalTok}[1]{#1}
\usepackage{graphicx,grffile}
\makeatletter
\def\maxwidth{\ifdim\Gin@nat@width>\linewidth\linewidth\else\Gin@nat@width\fi}
\def\maxheight{\ifdim\Gin@nat@height>\textheight\textheight\else\Gin@nat@height\fi}
\makeatother
% Scale images if necessary, so that they will not overflow the page
% margins by default, and it is still possible to overwrite the defaults
% using explicit options in \includegraphics[width, height, ...]{}
\setkeys{Gin}{width=\maxwidth,height=\maxheight,keepaspectratio}
\IfFileExists{parskip.sty}{%
\usepackage{parskip}
}{% else
\setlength{\parindent}{0pt}
\setlength{\parskip}{6pt plus 2pt minus 1pt}
}
\setlength{\emergencystretch}{3em}  % prevent overfull lines
\providecommand{\tightlist}{%
  \setlength{\itemsep}{0pt}\setlength{\parskip}{0pt}}
\setcounter{secnumdepth}{0}
% Redefines (sub)paragraphs to behave more like sections
\ifx\paragraph\undefined\else
\let\oldparagraph\paragraph
\renewcommand{\paragraph}[1]{\oldparagraph{#1}\mbox{}}
\fi
\ifx\subparagraph\undefined\else
\let\oldsubparagraph\subparagraph
\renewcommand{\subparagraph}[1]{\oldsubparagraph{#1}\mbox{}}
\fi

%%% Use protect on footnotes to avoid problems with footnotes in titles
\let\rmarkdownfootnote\footnote%
\def\footnote{\protect\rmarkdownfootnote}

%%% Change title format to be more compact
\usepackage{titling}

% Create subtitle command for use in maketitle
\newcommand{\subtitle}[1]{
  \posttitle{
    \begin{center}\large#1\end{center}
    }
}

\setlength{\droptitle}{-2em}

  \title{}
    \pretitle{\vspace{\droptitle}}
  \posttitle{}
    \author{}
    \preauthor{}\postauthor{}
    \date{}
    \predate{}\postdate{}
  

\begin{document}

\includegraphics{julia_analyse_files/figure-latex/unnamed-chunk-2-1.pdf}

ptratio: Schüler-Lehrer-Verhältnis nach Stadt. Durch die density
Funktion (schwarze Linie) kann man sehen und durch die Schiefe kann man
berechnen, dass die Verteilungsform eher linksschief ist (Schiefe
\textless{} 0) Anfangs angenommen es ist normalverteilt, aber man sieht,
dass das nicht so ist. blaue Linie = rnorm density; rote Linie = dnorm
Die Verteilung hat 2 Peaks: bei 14 bis 15 und 19 bis 20.5. 5 (bzw. 6)
Zahlen Zusammenfassung:

\begin{Shaded}
\begin{Highlighting}[]
  \KeywordTok{summary}\NormalTok{(Boston}\OperatorTok{$}\NormalTok{ptratio)}
\end{Highlighting}
\end{Shaded}

\begin{verbatim}
##    Min. 1st Qu.  Median    Mean 3rd Qu.    Max. 
##   12.60   17.40   19.05   18.46   20.20   22.00
\end{verbatim}

\includegraphics{julia_analyse_files/figure-latex/unnamed-chunk-4-1.pdf}

black: 1000 (Bk - 0,63) \^{} 2 wobei Bk der Schwarzanteil der Stadt ist.
Verteilung ist linksschief (Schiefe kleiner 0) und J-förmig.

\begin{Shaded}
\begin{Highlighting}[]
  \KeywordTok{skewness}\NormalTok{(Boston}\OperatorTok{$}\NormalTok{black)}
\end{Highlighting}
\end{Shaded}

\begin{verbatim}
## [1] -2.881798
\end{verbatim}

Annäherung an Paetro-Verteilung mit Intervall \([0, \infty]\) Die
Paetro-Verteilung (grüne Linie) hat hier die meisten Vorkommen bei den
höheren Werten Berechnung des Paramters \(\hat \xi\):

\begin{Shaded}
\begin{Highlighting}[]
  \KeywordTok{min}\NormalTok{(Boston}\OperatorTok{$}\NormalTok{black)}
\end{Highlighting}
\end{Shaded}

\begin{verbatim}
## [1] 0.32
\end{verbatim}

\includegraphics{julia_analyse_files/figure-latex/unnamed-chunk-7-1.pdf}
\includegraphics{julia_analyse_files/figure-latex/unnamed-chunk-8-1.pdf}

lstat: Prozentanteil der Bevölkerung mit niedriger Position in der
sozialen Hierarchie. Dh. schlechte menschliche Lebensumstände (z.B.:
wenig Bildung, keinen Schulabschluss, keine Ausbildung oder Studium,
geringes Einkommen, Migrationshintergrund, \ldots{}) Annahme: ptratio
ist normalverteilt die blaue und rote Linie zeigen, dass die Annahme
stimmt blaue Linie=rnorm density Parameter \(\mu\) und \(\sigma\):

\begin{Shaded}
\begin{Highlighting}[]
  \KeywordTok{mean}\NormalTok{(Boston}\OperatorTok{$}\NormalTok{lstat)}
\end{Highlighting}
\end{Shaded}

\begin{verbatim}
## [1] 12.65306
\end{verbatim}

\begin{Shaded}
\begin{Highlighting}[]
  \KeywordTok{sd}\NormalTok{(Boston}\OperatorTok{$}\NormalTok{lstat)}
\end{Highlighting}
\end{Shaded}

\begin{verbatim}
## [1] 7.141062
\end{verbatim}

\includegraphics{julia_analyse_files/figure-latex/unnamed-chunk-10-1.pdf}
\#\#\#\# medv

 mittlerer Wert von Wohneigentum in \$ 1000. Annäherung an
Normalverteilung, aber rechtsschief (Schiefe \textgreater{} 0) und
leptokurtisch (steilgipfelig) (Kurtosis \textgreater{} 3).

\begin{Shaded}
\begin{Highlighting}[]
  \KeywordTok{skewness}\NormalTok{(Boston}\OperatorTok{$}\NormalTok{medv)}
\end{Highlighting}
\end{Shaded}

\begin{verbatim}
## [1] 1.104811
\end{verbatim}

\begin{Shaded}
\begin{Highlighting}[]
  \KeywordTok{kurtosis}\NormalTok{(Boston}\OperatorTok{$}\NormalTok{medv)}
\end{Highlighting}
\end{Shaded}

\begin{verbatim}
## [1] 4.468629
\end{verbatim}

rote Linie = Normalverteilung, blaue Linie = Normalverteilung von medv
Werten \textless{} 50 (Ausreißer größer/gleich 50 weggeschnitten)

\subparagraph{Hypothesen:}\label{hypothesen}

Annahme: Das Bevölkerungswachstun in Bosten wird in den nächsten Jahren
weniger stark wachsen als in den Jahren zuvor und daher wird der
angenommen, dass der mittlere Wert des Whohnungseigentums sinkt und
deshhalb wird \(\mu\) mit 21 angenommen.\\
n = \(490 \geq\) 30; Daher kann die Z-Statistik verwendet werden.\\
\(\alpha = 5\)\%\\
\(\overline{x} = 21.6359184\)\\
\(s = 7.8653011\)\\
\(H_0: \mu = 21\)\\
\(H_1: \mu \gt 21\)

\begin{Shaded}
\begin{Highlighting}[]
\NormalTok{mu <-}\StringTok{ }\DecValTok{21}
\NormalTok{z <-}\StringTok{ }\NormalTok{(m}\OperatorTok{-}\NormalTok{mu)}\OperatorTok{/}\NormalTok{(s}\OperatorTok{/}\KeywordTok{sqrt}\NormalTok{(}\KeywordTok{length}\NormalTok{(x)))}
\NormalTok{p <-}\StringTok{ }\KeywordTok{pnorm}\NormalTok{(z, }\DataTypeTok{lower.tail =} \OtherTok{FALSE}\NormalTok{)}
\KeywordTok{show}\NormalTok{(p)}
\end{Highlighting}
\end{Shaded}

\begin{verbatim}
## [1] 0.03674981
\end{verbatim}

\begin{Shaded}
\begin{Highlighting}[]
\KeywordTok{show}\NormalTok{(p }\OperatorTok{<}\StringTok{ }\FloatTok{0.05}\NormalTok{)}
\end{Highlighting}
\end{Shaded}

\begin{verbatim}
## [1] TRUE
\end{verbatim}

Da der p-Wert unter 0.05 liegt, akzeptieren wir die Nullhypothese
\(H_0\).

\subparagraph{Kritischer Bereich:}\label{kritischer-bereich}

\begin{Shaded}
\begin{Highlighting}[]
\NormalTok{z <-}\StringTok{ }\KeywordTok{qnorm}\NormalTok{(}\FloatTok{0.025}\NormalTok{)}
\NormalTok{lower <-}\StringTok{ }\NormalTok{z}\OperatorTok{*}\NormalTok{(s}\OperatorTok{/}\KeywordTok{sqrt}\NormalTok{(}\KeywordTok{length}\NormalTok{(x)))}\OperatorTok{+}\NormalTok{mu}
\NormalTok{higher <-}\StringTok{ }\OperatorTok{-}\NormalTok{z}\OperatorTok{*}\NormalTok{(s}\OperatorTok{/}\KeywordTok{sqrt}\NormalTok{(}\KeywordTok{length}\NormalTok{(x)))}\OperatorTok{+}\NormalTok{mu}
\end{Highlighting}
\end{Shaded}

Der kritische Bereich für \(alpha = 0.05\) ist:
\(20.3035894 \leq \overline{x} \leq 21.6964106\). Das bedeutet, dass
\(H_0\) gültig ist, solange der wahre Mittelwert in diesem Bereich
liegt.


\end{document}
